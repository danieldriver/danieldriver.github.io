% !TEX encoding = UTF-8 Unicode
% !TEX TS-program = XeLaTeX

\documentclass[11pt,twocolumn]{article}
\usepackage[%
%	showframe,
	letterpaper,
	scale=0.8,
	columnsep=0.25in]{geometry}

% Typography
\usepackage{fontspec,realscripts}
\defaultfontfeatures{Ligatures=TeX,Numbers=OldStyle}
\setromanfont{Arno Pro}
\setsansfont[Scale=MatchLowercase]{Scala Sans Pro}
\setmonofont[Scale=MatchLowercase]{Menlo}
\usepackage{sectsty}
\allsectionsfont{\sffamily}
\frenchspacing
\usepackage[normalem]{ulem} % provides strikethrough with \sout{}

% Tables, links, metadata, poems
\usepackage{array}     % allow insertions of column styling with >{}
\usepackage{booktabs}  % elegant horizontal rules in tables
\usepackage{graphicx}
\usepackage{verse}

% Issue list
\usepackage{enumitem}
\newlist{issues}{itemize}{2}
\setlist[issues]{%
	parsep=0em,
	itemsep=0ex,
	label=\underline{\hspace{1.5em}}}

% Title
\title{Guidelines for Papers and Other Written Assignments}
\author{Daniel R. Driver, PhD\thanks{I would like to acknowledge two
teachers who might recognize echoes of their advice here. Jeffry C.
Davis introduced me to composition theory, practice, and pedagogy, and
Nathan MacDonald helped me see the use of setting an effective book
review assignment.}}
\date{}

\begin{document}
\maketitle

\section{Make an Argument}

Academic writing aims to persuade. In my courses, therefore, you are
asked to craft thoughtful, defensible arguments. Whether you are
summarizing an article, conducting a book review, interpreting an
ancient text, outlining a theological position, organizing historical
evidence, or advancing a thesis in a major paper, you should be certain
that you have a point to make, and that your writing serves that point
effectively. Different assignments call for arguments to be mounted in
different ways, but in virtually every case your submission should have
a clear thesis statement that is well supported by the rest of the
piece. Your readers should be able to identify it easily, not least
because it takes the form of a single, grammatical sentence. In the best
cases it will even be elegant. As Nicholas Lash writes, “God’s beauty is
not well served by ugly prose.”\footnote{Nicholas Lash, \emph{The
Beginning and the End of “Religion”} (Cambridge: Cambridge University
Press, 1996), 122.}

What makes a thesis strong? Arguments can take many forms, but a good
thesis has three main characteristics: it must be \textbf{restricted},
\textbf{unified}, and \textbf{precise}.\footnote{Margot Northey,
Bradford A. Anderson, and Joel N. Lohr, \emph{Making Sense in Religious
Studies: A Student's Guide to Research and Writing}, 3rd ed. (Don Mills,
ON: Oxford University Press, 2019), 64–66.}

First, to be \textbf{restricted}, a thesis must limit the scope of an
essay to what can be discussed in detail within the bounds of the
paper's expected length. Whole monographs have been written on single
psalms. It is simply not realistic to think that you can prove something
about the entire Book of Psalms in twenty pages. A good thesis is narrow
in focus. For example: “The divine warrior motif in Psalm 24:7–10 finds
new resonance in later Christian liturgies of Christ's Harrowing of
Hell, as shown by the placement of movement 33 in the libretto of
Handel's \emph{Messiah}.”

Second, a thesis must be \textbf{unified}; in other words, it must
express only one idea. Consider the following sentence: “Sabbath keeping
has different rationales in the Old Testament, is radically transformed
by Jesus in the New Testament, and yet interest in the topic seems to be
growing as Christians struggle to find rest and renewal in their busy
lives.” The problem with this thesis is that it commits the writer to
several different topics: (1) rationales for sabbath observance in
ancient Israel, (2) the transformation of sabbath practices in the early
Church, and (3) contemporary accounts of spirituality that may have as
much in common with therapeutic self-help literature as with traditional
religious devotion. When a thesis uses a coordinating sentence,
containing two or more parts, it is important that all of the parts are
closely related so that the focus of the paper is not diffused.

Third, a good thesis is \textbf{precise}, meaning that when it is read,
it can only have one interpretation. Vague words should be avoided. And
abstract words may be problematic, unless the paper is dealing with an
abstract issue. For example, a theology paper dealing with Thomist
metaphysics may require more abstract language than an exegetical paper
dealing with manuscript evidence from the Dead Sea Scrolls. Generally,
try to make the thesis as clear and specific as possible, fleeing from
words that express nondescript meaning, such as \emph{unusual},
\emph{interesting}, \emph{inspiring}, or \emph{important}, to name a
few.

Finally, a thesis sentence is most often expressed in a single sentence,
and is generally found at the end of a well developed introductory
paragraph. This rule is not set in stone but usually works well because
it helps the reader to make an easy transition into the body of the
paper. Remember, while you do have some flexibility in the way you craft
your thesis sentence, without a thesis that is restricted, unified, and
precise, it is nearly impossible to have a good paper.

\vfill

{\noindent\itshape In class or at home: rewrite the poor thesis in the
paragraph on unity, above. Make it restricted, unified, and precise.}

\section{Consider the Specific Assignment}

Always refer to the syllabus for the exact parameters of your
assignments in a given course. For general help with some common writing
assignments in my courses, see below.

\subsection{The Forum Post}

Does a 250-word forum post need to have a thesis? Not necessarily. A
class forum is a fine place to experiment with creative approaches to
big questions, so please feel free to try something unusual. Responses
could take the form of a letter (to the editor, say), a dialogue (after
Plato), an imitation of a literary genre (including those of scripture),
a children's sermon, or anything else that occurs to you as a productive
way to engage the question. Then again, the venue also gives you an
audience of peers with which to hone your persuasive writing skills.
Even if you are brimming with creative ideas for prompt responses, take
at least four opportunities over the semester to apply that energy to
the humble paragraph. Paragraphs are the building blocks of essays, and
it takes experience to know how to identify a paragraph-sized idea and
then nurture it into a form that can help support a larger argument. Use
the forum to practice. Learn to develop a robust unit of thought
governed by a clear topic sentence, which is a thesis in miniature. How?
Compose a single, well-ordered paragraph in answer to four or more
discussion prompts this term.

\subsection{The Review (Book or Article)}

Literature review can facilitate critical reflection on a significant
work of biblical interpretation or theology, but note well: a book
review is not the same thing as a book report. A report simply
summarizes the contents of a work, as if the only goal were to prove
that you read it, whereas a review is a focused exercise in critical
analysis and evaluation. In very limited space, this kind of paper
should first explain the message of a book or article, and then present
an assessment of its significance. It should have the following
elements:

\begin{enumerate}

\item full bibliographic details about the work under review. Let the
reference follow SBL Style. Parenthetical citations may follow, thus:
“quote” (79).

\item an introductory hook (a phatic or attention-getting device). It
will be quite short compared to the introductions most other papers
require.

\item a summary of the work and its arguments. You should describe the
main argument in your own words and through the judicious use of quotes.
Try to be objective (would the author recognize your summary of their
work? is it fair?) and insightful (do you capture the work's central
points?). Please avoid critical comment at this stage.

\item an analysis and evaluation of the work, possibly through
discussion of a representative example. If the scholar's focus is
exegetical, examine a characteristic instance of biblical
interpretation. Read the biblical passage being interpreted, and ask
yourself the following sorts of questions: Is the interpretation
insightful? Does it help you understand the biblical text? Does it
explain every feature of it? What does it omit or pass over? Why? Are
there any problems with the interpretation? Does the biblical text aid
the scholar’s wider argument? Alternately, if the work is not especially
exegetical, consider the list of questions suggested by Northey et
al.\footnote{Northey, Anderson, and Lohr, \emph{Making Sense}, 85–86.}
Critical comments should be thoughtful and measured.

\item optionally, a brief statement of your personal response to the
work.

\item an apt conclusion. The final paragraph is often a good place to
put a thesis statement, since the reviewed work must be understood
before you can make an argument about it.

\end{enumerate}

Unless the professor or syllabus states otherwise, about half of the
paper (50\%) should be evaluative. If the assignment calls for 1,000
words, you might choose to distribute them as follows (the bibliographic
citation at the top does not count against your word limit).

\begin{table}[htbp]
  \centering
  {\begin{tabular}{lr}
    \toprule
    Introduction & 80  \\
    Summary      & 400 \\
    Evaluation   & 400 \\
    Conclusion   & 120 \\
    \bottomrule
  \end{tabular}}
%  \caption{Sample proportions for a review}
%  \label{review}
\end{table}

\noindent Mere plot summary is not required. In the case of an assigned
text, unlike a review that might appear in print, you can assume that
your audience is already at least somewhat familiar with the work in
question.

\subsection{The Exegetical Essay}

The basic task of an exegetical essay is to identify a suitable passage
from the Bible (perhaps three to six verses), and to offer a reading
of it that supports a thesis.

The biblical exegete needs a plan of attack. Unfortunately, I know of no
method that applies universally. I took a few courses on the Bible when
I was an undergraduate. In one I remember being presented with an
authoritative list of steps for exegesis – I believe there were twelve
steps in all. At the time, as a student of English Literature, the
hermeneutical program struck me as artless. It was far too formulaic.
Now that I have looked into the matter more thoroughly, I have become
convinced that all such schemes are artificial. They might serve a
purpose for a time and a place, but they cannot guarantee excellent
results, even in the limited contexts in which they arise. This problem
becomes more obvious when one studies the history of the Bible's
interpretation. Some methods that worked for Origen or Jerome in one
situation could not be followed by Diodore or Theodore in another. And
Augustine's famous treatise on the interpretation of scripture, called
\emph{On Christian Doctrine}, is still worth studying, but it does not
resolve all the interpretive conundrums that readers of the Bible have
faced ever since. What is one to do? If you find yourself glowing with
confidence about your favourite approach to the Bible, the first step
might be to take a dose of humility. Walk through a world class
theological research library, if you get the chance, and marvel at just
how much has been written about the Bible over the centuries. If, on the
other hand, you find yourself daunted by the challenge, I commend the
steps outlined in \emph{Making Sense in Religious Studies} as a
reasonable and practical place to start.\footnote{For details on each of
these eight steps see Northey, Anderson, and Lohr, \emph{Making Sense},
105–112.}

\begin{enumerate}

\item Find a text of suitable length.

\item Translate the passage or read it through in multiple translations.

\item Determine the genre.

\item Conduct literary analysis.

\item Examine the historical context.

\item Examine the compositional history (when possible).

\item Consult secondary literature.

\item Iterate. Revisit the above steps.

\end{enumerate}

Note, too, that the thesis of an exegetical essay should have an organic
relationship to the text it seeks to describe. If you only find in a
passage something you already thought was there, the exercise has
failed. Practically speaking, this means that the final version of your
thesis statement may be the last thing you are able to write. However, a
lucid statement of it still belongs in the paper's introduction. Lead
off by presenting the results of your study of a biblical text, and then
take the reader through the exegetical spade work that got you there. It
can even be a virtue if, in the interest of honesty, you point out some
limits in your reading. The goal is not to override but to clarify the
text.

\subsection{The Thesis}

A thesis proper, by which I mean not just a sentence but a total
argument, is often a culminating work in a degree program. I regularly
set a question that calls for a final thesis as the culminating
assignment of a course as well. It requires all the things that have
been described above: a sound thesis statement, mature and topical
supporting paragraphs, competence in secondary literature, and a solid
exegetical foundation. There is considerable freedom in how these
elements may be combined, which means there is scope for creativity.
Then again, the argumentative essay is an established genre. While a
final thesis for a graduate-level course should take you well beyond the
classic formula of a five-paragraph essay, it can help to recall the
core elements of that exercise, which are (1) an introductory paragraph
with a concise thesis statement, (2) body paragraphs that support the
paper's thesis with evidence, analysis, and argument, and (3) a
concluding paragraph that synthesizes the total argument in view of its
evidential basis.

%[\emph{This handout, too, is a work in progress! TK\ldots}]

How might you approach the job? My main advice is to take it in stages,
separated by several good nights of sleep. Leaving most of the work to
the day before a deadline is no recipe for success. Give yourself as
much time as you can to revise your work, knowing that writing is a
rigorous process. Write. Reflect. Rewrite. Edit. Repeat as necessary. It
takes real effort to make prose appear effortless.

%\section{Write, Reflect, Rewrite, Edit}
%
%\subsection{Write Shitty Rough Drafts}
%
%\subsection{Take a Nap}
%
%
%Or, read a poem, such as the following one by Scott
%Cairns.\footnote{First printed in \emph{The Theology of Doubt} (1985),
%Cairns's “On Slow Learning” is reprinted in \emph{Compass of Affection:
%Poems New and Selected} (Brewster, MA: Paraclete Press, 2006), 5.}
%
%\poemtitle{\sffamily On Slow Learning}
%\settowidth{\versewidth}{how difficult paper training can be}
%\begin{verse}[\versewidth]
%If you have ever owned\\
%a tortoise, you already know\\
%how terribly difficult\\
%paper training can be\\
%for some pets.
%
%Even if you get so far\\
%as to instill in your tortoise\\
%the value of achieving the paper,\\
%there remains one obstacle---\\
%your tortoise’s intrinsic sloth.
%
%Even a well-intentioned tortoise\\
%may find himself, in his journeys,\\
%to be painfully far from the mark.
%
%Failing, your tortoise may shy away\\
%for weeks within his shell, utterly\\
%ashamed, or looking up with tiny,\\
%wet eyes might offer an honest shrug.\\
%Forgive him.
%\end{verse}
%
%\subsection{Write and Rewrite}
%
%\subsection{Try the Paramedic Method}
%
%See the Online Writing Lab (OWL) at Purdue University for details.
%Search for “Paramedic Method” or go to:
%https://owl.english.purdue.edu/owl/resource/635/01/

\onecolumn
\thispagestyle{empty}
\section*{Dr. Driver's Paper Analysis and Assessment}

\subsection*{Surface-level Issues}

\begin{issues}

\item Format Violations: Follow \emph{The SBL Handbook of Style}, 2nd ed. (which supplements \emph{The Chicago Manual of Style})
\item Awkward Sentence (\emph{awk}): Wordy/Strange meaning/Clumsy arrangement
\item Fragment (\emph{frag}): Incomplete sentence, usually missing a verb
\item Run-on (\emph{RO}): Two independent clauses have been fused together
\item Comma Splice (\emph{cs}): Comma used in the place of a period, or without a coordinating conjunction
\item Predication Fault (\emph{pred}): The grammatical subject is disconnected from what follows (the predicate)
\item Agreement Error (\emph{agr/ref}): Subject-verb or pronoun-antecedent
\item Verb Error (\emph{vb}): The wrong verb tense is used, or it changes without good reason
\item Not Parallel (\emph{//}): Sentence elements incorrectly linked; they must be coordinated
\item Passive Voice (\emph{PV}): \sout{Active voice is preferred} Write with active verbs
\item Redundant (\emph{rep}): Do not repeat words or ideas unnecessarily
\item Spelling/Wrong Word (\emph{sp/ww}): Check the dictionary for standard usage
\item Punctuation (\emph{P}): Observe the standard rules; consult Ch. 17 in \emph{Making Sense in Religious Studies}
\item Clichés: Overused phrases do not engage the reader with fresh expression
\item Miscellaneous:

\end{issues}

\subsection*{Deeper-level Issues}

\begin{issues}

\item Thesis Problem (\emph{T}): Missing/Unclear/Weak/Nonspecific/Awkward/Not proven by the paper
\item Introduction Weak: Phatic device/Lack of clarity/Lack of development
\item Poor Organization: Ideas scattered/Illogical order/Irrelevant material
\item Inadequate Explanation: Need more: Details/Descriptions/Examples/Illustrations/Quotes
\item Paragraphs (\emph{\P\P}) Lacking: Unity/Coherence/Development
\item Transitions (\emph{trans.}): Paragraphs change abruptly/Ideas are not clearly connected to one another
\item Sentence Style Too Similar: Vary the types and lengths more often
\item Conclusion: Does not bring the essay to a satisfying or natural close
\item Persuasive Caliber: Lacks logic/Lacks passion/Lacks credibility/Lacks clarity/Lacks interpretation
\item Reasoning Flaws: Overgeneralizations/Unqualified statements/Lack of proof or support
\item Audience/Occasion: The writing does not engage a particular group or person, or it lacks purpose
\item Diction (\emph{D}): Too colloquial/Vague/Wordy/Jargon heavy; balance Latinate and Anglo-Saxon words
\item Tone: Too personal/Too informal/Too formal/Biased language; maintain an appropriate style
\item Miscellaneous:

\end{issues}

\subsection*{Final Comments (Note that a + sign indicates a strength of this paper, while a \textminus\ sign indicates a critique)}


%\end{minipage}

\end{document}
